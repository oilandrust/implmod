\title{Master Project - Real Time Rendering of skeletal structures - Notes on Convolution Surfaces}
\author{
        %\large
        \textsc{Olivier Rouiller - s090842}
        \mbox{}\\ %
        Department of Informatics and Mathematical Modelling\\
        Technical University of Denemark\\
        \mbox{}\\ %
}
\date{\today}
\documentclass[11pt]{article}
%\documentclass{acmconf}

\usepackage[paper=a4paper,dvips,top=1.5cm,left=1.5cm,right=1.5cm,
    foot=1cm,bottom=1.5cm]{geometry}

\usepackage{times}
%\usepackage{graphicx}
\usepackage[fleqn]{amsmath}
\usepackage{amsfonts}
\usepackage{amssymb}
\usepackage{amsthm}
\usepackage{amsopn}
\usepackage{xspace}
\usepackage{array}
\usepackage{epsfig}

\numberwithin{figure}{section}

\newcommand\CC{\Lang{\mbox{C++}}\xspace}
\newcommand\Lang[1]{\textsc{#1}}
\newcommand{\kw}[1]{\texttt{\textbf{#1}}}
\newcommand{\cd}[1]{\texttt{#1}}

\newcommand\Naturals{\ensuremath{\mathbb{N}}\xspace}
\newcommand\Integers{\ensuremath{\mathbb{Z}}\xspace}
\newcommand\Rationals{\ensuremath{\mathbb{Q}}\xspace}
\newcommand\Reals{\ensuremath{\mathbb{R}}\xspace}
\newcommand\Complex{\ensuremath{\mathbb{C}}\xspace}

\newcommand\norm[1]{\ensuremath{\lVert#1\rVert}}
\newcommand\abs[1]{\ensuremath{\lvert#1\rvert}}
\newcommand\ceil[1]{\ensuremath{\lceil#1\rceil}}
\newcommand\floor[1]{\ensuremath{\lfloor#1\rfloor}}
\newcommand\set[1]{\ensuremath{\{#1\}}}
\newcommand\angular[1]{\ensuremath{\langle#1\rangle}}

\newcommand\Norm[1]{\ensuremath{\left\lVert#1\right\rVert}}
\newcommand\Abs[1]{\ensuremath{\left\lvert#1\right\rvert}}
\newcommand\Ceil[1]{\ensuremath{\left\lceil#1\right\rceil}}
\newcommand\Floor[1]{\ensuremath{\left\lfloor#1\right\rfloor}}
\newcommand\Set[1]{\ensuremath{\left\{#1\right\}}}
\newcommand\Angular[1]{\ensuremath{\left\langle#1\right\rangle}}

\newcommand{\LOOM}{\ensuremath{\cal{LOOM}}\xspace}
\newcommand{\PolyTOIL}{\textbf{PolyTOIL}\xspace}

\newtheorem{theorem}{Theorem}[section]
\newtheorem{definition}[theorem]{Definition}
\newtheorem{lemma}[theorem]{Lemma}
\newtheorem{corollary}[theorem]{Corollary}
\newtheorem{fact}[theorem]{Fact}
\newtheorem{example}[theorem]{Example}

\newcommand\Cls[1]{\textsf{#1}}
\newcommand\Fig[1]{Figure~\ref{Figure:#1}}

\usepackage{labels} %
\usepackage{equation}
\usepackage{prog2tex}

\newenvironment{excerpt}{\begin{quote}\begin{minipage}\textwidth}{\end{minipage}\end{quote}}

\setcounter{topnumber}{0}
\setcounter{bottomnumber}{0}
\setcounter{totalnumber}{20}
\renewcommand{\textfraction}{0.01}

\begin{document}

\maketitle

\Section[simple]{Convolution surfaces}

Convolution surfaces generalizes metaballs.
Iso levelset of potential fields $f(p)=g(p)\times h(p)= \int_{R^3} \! g(r)h(p-r) \, \mathrm{d}r. $
Where $g$ is the geometric potential field, defined by the underlying geometry, $g=1$ where the geometry is defined, $g=0$ elsewhere.
$h$ is a convolution kernel, typicaly distance functions.

Example of kernels (see thesis on cinvolution surfaces)

\begin{itemize}

\item{Cauchy function}
\begin{center}
$h(r)=1/(1+s^2r^2), r > 0$
\end{center}

\item{Gaussian function}
\begin{center}
$h(r)=exp(-a^2r^2), r > 0$
\end{center}

\item{Inverse function}
\begin{center}
$h(r)=1/r, r > 0$
\end{center}

\item{Inverse squared function}
\begin{center}
$h(r)=1/r^2, r > 0$
\end{center}

\item{Soft objects}
\begin{center}
$h(r)=1-\frac{4}{9}r^6+\frac{17}{9}r^4-\frac{22}{9}r^2, r < 1, 0, r> 1$
\end{center}

\item{Metaballs}
\begin{center}
$h(r)=1-3r^2, 0 <r < 1/3, \frac{3}{2}(1-r)^2, 1/3< r < 1, 0, r> 1$
\end{center}

\item{W-shaped quartic polynomial}
\begin{center}
$h(r)= (1-r^2)^2,  r < 1, 0, r> 1$
\end{center}

\end{itemize}



\bibliography{references.bib} 
\bibliographystyle{abbrv}

\end{document}
