\title{Master Project - Real Time Rendering of skeletal structures - Motivations Goals and Plan}
\author{
        %\large
        \textsc{Olivier Rouiller - s090842}
        \mbox{}\\ %
        Department of Informatics and Mathematical Modelling\\
        Technical University of Denemark\\
        \mbox{}\\ %
}
\date{\today}
\documentclass[11pt]{article}
%\documentclass{acmconf}

\usepackage[paper=a4paper,dvips,top=1.5cm,left=1.5cm,right=1.5cm,
    foot=1cm,bottom=1.5cm]{geometry}

\usepackage{times}
%\usepackage{graphicx}
\usepackage[fleqn]{amsmath}
\usepackage{amsfonts}
\usepackage{amssymb}
\usepackage{amsthm}
\usepackage{amsopn}
\usepackage{xspace}
\usepackage{array}
\usepackage{epsfig}
\usepackage{cite}
\usepackage{pdfpages} 


\numberwithin{figure}{section}

\newcommand\CC{\Lang{\mbox{C++}}\xspace}
\newcommand\Lang[1]{\textsc{#1}}
\newcommand{\kw}[1]{\texttt{\textbf{#1}}}
\newcommand{\cd}[1]{\texttt{#1}}

\newcommand\Naturals{\ensuremath{\mathbb{N}}\xspace}
\newcommand\Integers{\ensuremath{\mathbb{Z}}\xspace}
\newcommand\Rationals{\ensuremath{\mathbb{Q}}\xspace}
\newcommand\Reals{\ensuremath{\mathbb{R}}\xspace}
\newcommand\Complex{\ensuremath{\mathbb{C}}\xspace}

\newcommand\norm[1]{\ensuremath{\lVert#1\rVert}}
\newcommand\abs[1]{\ensuremath{\lvert#1\rvert}}
\newcommand\ceil[1]{\ensuremath{\lceil#1\rceil}}
\newcommand\floor[1]{\ensuremath{\lfloor#1\rfloor}}
\newcommand\set[1]{\ensuremath{\{#1\}}}
\newcommand\angular[1]{\ensuremath{\langle#1\rangle}}

\newcommand\Norm[1]{\ensuremath{\left\lVert#1\right\rVert}}
\newcommand\Abs[1]{\ensuremath{\left\lvert#1\right\rvert}}
\newcommand\Ceil[1]{\ensuremath{\left\lceil#1\right\rceil}}
\newcommand\Floor[1]{\ensuremath{\left\lfloor#1\right\rfloor}}
\newcommand\Set[1]{\ensuremath{\left\{#1\right\}}}
\newcommand\Angular[1]{\ensuremath{\left\langle#1\right\rangle}}

\newcommand{\LOOM}{\ensuremath{\cal{LOOM}}\xspace}
\newcommand{\PolyTOIL}{\textbf{PolyTOIL}\xspace}

\newtheorem{theorem}{Theorem}[section]
\newtheorem{definition}[theorem]{Definition}
\newtheorem{lemma}[theorem]{Lemma}
\newtheorem{corollary}[theorem]{Corollary}
\newtheorem{fact}[theorem]{Fact}
\newtheorem{example}[theorem]{Example}

\newcommand\Cls[1]{\textsf{#1}}
\newcommand\Fig[1]{Figure~\ref{Figure:#1}}

\usepackage{labels} %
\usepackage{equation}
\usepackage{prog2tex}

\newenvironment{excerpt}{\begin{quote}\begin{minipage}\textwidth}{\end{minipage}\end{quote}}

\setcounter{topnumber}{0}
\setcounter{bottomnumber}{0}
\setcounter{totalnumber}{20}
\renewcommand{\textfraction}{0.01}

\begin{document}

\maketitle

\Section[simple]{Our problem}

Design an algorithm to raytrace efficiently a implicit surface defined by a skeleton.
\begin{itemize}

\item{} The algorithm must raytrace the surface efficiently, if possible as fast as rasterizing a mesh.
\item The algorithm must get the entire surface.
\item The algorithm must allow dynamic skeleton.
\item The algorithm must be integrable in a classical rendering pipeline (deferred or direct?).
\item The algorithm must allow as many effect as possible.

\end{itemize}

How to speed up the ray-tracing?

\begin{itemize}

\item{} Shoot ray only where there is a surface intersection. Involves spatial subdivision techniques such as BVH, Octree, ...
\item Speed up intersection. Choice of root finding algo, using geometry.
\item Evaluate only the primitives that contributes at the point of evaluation. spatial sorting.

\end{itemize}

\Section[simple]{Techniques for raytracing Implicit surfaces}

Hart gives a good intro to different techniques in Ray-Tracing implicit surfaces.

\subsection{Polynomial Root Solving}

Works for algebraic surfaces (in our scope).

First work seems to be from Hanrahan \cite{Hanrahan:1983:RTA:800059.801136}. His method used "a symbolic
algebra system to automatically derive the
equation of intersection between the ray
and the surface and then solves this equation
using an exact polynomial root finding
algorithm". This implies converting the space function of the surface to a univariate function along the ray and then solve this polynomial equation.
Having the coefficients of the ray polynomial, numerical methods for root finding are used.


The same approach is used in \cite{Loop:2006:RGR:1179352.1141939} but the univariate ray polynomial is computed using Bezier tetrahedra.

\subsection{Interval analysis}

A common method to find root of a function along ray is to use interval analysis, isolating the root in an interval then refining this interval to approximate the root. \cite{Mitchell:1990:RRI:93267.93276}.

\subsection{Lipschitz methods}

Interval analysis can be improved by using the local Lipschitz constant of the function at each iteration.

\subsection{Sphere Tracing}

Another method to raytrace an implicit surface is to use sphere tracing \cite{springerlink:10.1007/s003710050084}. 
This method does not involve root finding but finds the first intersection of the ray with the surface.
This relies on knowing the distance of the current point on the ray to the surface. If this distance is less than $\epsilon$, we found the intersection, otherwise we know that we can move forward with the distance to the surface. It seems that in the case of convolution surfaces, we know this distance.

\clearpage

\bibliography{../../references} 
\bibliographystyle{alpha}

\end{document}
